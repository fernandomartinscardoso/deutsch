% Appendix - Basics of Phonetics and Phonology

\fancyhead[RO]{\slshape Appendix A}  % Mark on right odd pages
\fancyhead[LE]{\slshape Deutschkurs}     % Mark on left even pages
\chapter{Basics of Phonetics and Phonology}
\label{app:phonetics_phonology}

\section*{Syllables}
\addcontentsline{toc}{section}{Syllables}
\label{sec:syllables}

A syllable is a unit of pronunciation voiced without interruption, close to a single sound. All words are made from at least one syllable \cite{wiki_syllable}.

According to Ramoo \cite{ramoo2021}, while phonemes are the smallest units of sound, we don't actually speak in phonemes. If you say the word ``cat'' \textipa{[k{\ae}t]} and record it, you won't be able to break it into three units of \textipa{[k]}, \textipa{[\ae]} and \textipa{[t]}. 

Therefore, the smallest unit of articulation is not the phoneme but rather the \textit{syllable}. Most native speakers of a language will know how many syllables are in a word in their language. You can try this in English by saying a word slowly. For example, the word ``elephant'' has three syllables: \textipa{["el.I.f@nt]}. As seen in Figure \ref{fig:syllable}, all syllables must have a mandatory \textit{nucleus} or peak. This is usually a vowel. Some languages can also have a syllabic consonant as a nucleus of a syllable as in the English word ``button'' \textipa{[b2t.\s{n}]} where there are two syllables \textipa{[b2t]} and \textipa{[\s{n}]}. You can see that the second syllable has no vowels but a syllabic \textipa{[\s{n}]} as the nucleus.

\begin{figure}[H]
    \centering
    \includegraphics[width=0.4\textwidth]{figures/syllable_structure.png}
    \caption{\small Syllable structure \cite{ramoo2021}.\normalsize}
    \label{fig:syllable}
\end{figure}

Consonants that come before the nucleus of a syllable are know as \textit{onsets} and those that come after it are called \textit{codas}. The nucleus and coda of a syllable form a group called a \textit{rime}. These onsets and codas can be complicated or simple depending on what is allowed in a language. English allows up to three consonants in the onset and at least as much in the coda. Consider the word ``twelfths'' \textipa{[twElfTs]}. It has two consonants in the onset and four consonants in the coda. Generally, the onset is more restricted in what consonants are allowed \cite{ramoo2021}.

\subsection*{Properties of Syllables}
\addcontentsline{toc}{subsection}{Properties of Syllables}
\label{ssec:syllables_properties}

If a syllable ends with a consonant, it is called a \textit{closed syllable}. If a syllable ends with a vowel, it is called an \textit{open syllable} \cite{wiki_syllable}. In summary, closed syllables have coda consonants.

The \textit{stress} or \textit{accent} is the relative emphasis or prominence given to a certain syllable in a word or to a certain word in a phrase or sentence \cite{wiki_stress}. But this accent must not be confused with the sociolinguistic meaning of accent, which is a way of pronouncing a language that is distinctive to a country, area, social class, or individual \cite{wiki_accent}.