% Appendix - Basics of Phonetics and Phonology

\fancyhead[RO]{\slshape Appendix A}  % Mark on right odd pages
\fancyhead[LE]{\slshape Deutschkurs}     % Mark on left even pages
\chapter{Basics of Phonetics and Phonology}
\label{app:phonetics_phonology}

\section{Syllables}
\addcontentsline{toc}{section}{Syllables}
\label{sec:syllables}

A syllable is a unit of pronunciation voiced without interruption, close to a single sound. All words are made from at least one syllable \cite{wiki_syllable}.

According to Ramoo \cite{ramoo2021}, while phonemes are the smallest units of sound, we don't actually speak in phonemes. If you say the word ``cat'' \textipa{[k{\ae}t]} and record it, you won't be able to break it into three units of \textipa{[k]}, \textipa{[\ae]} and \textipa{[t]}. 

Therefore, the smallest unit of articulation is not the phoneme but rather the \textit{syllable}. Most native speakers of a language will know how many syllables are in a word in their language. You can try this in English by saying a word slowly. For example, the word ``elephant'' has three syllables: \textipa{["el.I.f@nt]}. As seen in Figure \ref{fig:syllable}, all syllables must have a mandatory \textit{nucleus} or peak. This is usually a vowel. Some languages can also have a syllabic consonant as a nucleus of a syllable as in the English word ``button'' \textipa{[b2t.\s{n}]} where there are two syllables \textipa{[b2t]} and \textipa{[\s{n}]}. You can see that the second syllable has no vowels but a syllabic \textipa{[\s{n}]} as the nucleus.

\begin{figure}[H]
    \centering
    \includegraphics[width=0.4\textwidth]{figures/syllable_structure.png}
    \caption{\small Syllable structure \cite{ramoo2021}.\normalsize}
    \label{fig:syllable}
\end{figure}

Consonants that come before the nucleus of a syllable are know as \textit{onsets} and those that come after it are called \textit{codas}. The nucleus and coda of a syllable form a group called a \textit{rime}. These onsets and codas can be complicated or simple depending on what is allowed in a language. English allows up to three consonants in the onset and at least as much in the coda. Consider the word ``twelfths'' \textipa{[twElfTs]}. It has two consonants in the onset and four consonants in the coda. Generally, the onset is more restricted in what consonants are allowed \cite{ramoo2021}.

\subsection{Properties of Syllables}
\addcontentsline{toc}{subsection}{Properties of Syllables}
\label{ssec:syllables_properties}

If a syllable ends with a consonant, it is called a \textit{closed syllable}. If a syllable ends with a vowel, it is called an \textit{open syllable} \cite{wiki_syllable}. In summary, closed syllables have coda consonants.

The \textit{stress} or \textit{accent} is the relative emphasis or prominence given to a certain syllable in a word or to a certain word in a phrase or sentence \cite{wiki_stress}. But this accent must not be confused with the sociolinguistic meaning of accent, which is a way of pronouncing a language that is distinctive to a country, area, social class, or individual \cite{wiki_accent}.

\section{Voiced and Voiceless Sounds}
\addcontentsline{toc}{section}{Voiced and Voiceless Sounds}
\label{sec:voiced_and_voiceless}

Our vocal cords, or vocal folds, are two muscular bands inside our voice box that produce the sound of our voice. The way we use them to pronounce certain sounds defines if the sound is \textit{voiced} or \textit{voiceless}.

The difference is whether the vocal cords vibrate: voiced sounds are produced with vibrating vocal cords, while voiceless sounds are produced without vibration. You can feel this difference by placing your hand on your throat; a "buzzing" sensation indicates a voiced sound, whereas a lack of vibration indicates a voiceless sound. All vowels are voiced, and some consonants are voiced, while others are voiceless. 

\begin{table}[H]
\centering
\begin{tabular}{|l|l|l|}
\hline
\textbf{Elements} & \textbf{Voiced Sounds}                                                                                          & \textbf{Voiceless Sounds}                                                                                        \\ \hline
\hline
Vocal Cords       & Vibrate                                                                                                         & Do not vibrate                                                                                                  \\ \hline
Sensation         & Buzzing or humming in the throat                                                                                & No vibration in the throat                                                                                      \\ \hline
Mechanism         & \begin{tabular}[c]{@{}l@{}}Air from the lungs closes the vocal \\ cords and causes them to vibrate\end{tabular} & \begin{tabular}[c]{@{}l@{}}Air flows freely through the mouth \\ without the vocal cords vibrating\end{tabular} \\ \hline
Example           & \textipa{[b]} in ``bat''                                                                                        & \textipa{[p]} in ``pat''                                                                                        \\ \hline
Other Cases       & \textipa{[z]}, \textipa{[v]}, \textipa{[m]}                                                                     & \textipa{[k]}, \textipa{[s]}, \textipa{[t]}                                                                     \\ \hline
\end{tabular}
\caption{\small Voiced and voiceless sounds.\normalsize}
\label{tab:voiced_and_voiceless}
\end{table}

\section{Short and Long Vowels}
\addcontentsline{toc}{section}{Short and Long Vowels}
\label{sec:short_and_long_vowels}

Short and long vowels differ in the length and quality of their sound: short vowels are quick sounds like \textipa{[a]} in ``cat'', while long vowels are drawn out and sound like the letter's name, such as \textipa{[a]} in ``cake''. In English, long vowels are often created by a ``silent e'' at the end of a word or by having two vowels together, and their sounds are held for a longer duration than short vowels.

\begin{table}[H]
\centering
\begin{tabular}{|l|l|}
\hline
\textbf{Short Vowel}                                  & \textbf{Long Vowel}                                     \\ \hline
\hline
\textipa{[\ae]} as in \textbf{a}pple, c\textbf{a}t    & \textipa{[{\;E}I]} as in \textbf{a}corn, c\textbf{a}ke  \\ \hline
\textipa{[E]} as in \textbf{e}gg, b\textbf{e}d        & \textipa{[I:]} as in \textbf{e}ve, sc\textbf{e}ne       \\ \hline
\textipa{[I]} as in \textbf{i}nk, s\textbf{i}t        & \textipa{[{\;A}I]} as in \textbf{i}ce, b\textbf{i}ke    \\ \hline
\textipa{[A]} as in \textbf{o}ctopus, m\textbf{o}p    & \textipa{[{\;O}U]} as in \textbf{o}pen, n\textbf{o}te   \\ \hline
\textipa{[2]} as in \textbf{u}mbrella, c\textbf{u}t   & \textipa{[ju:]} as in \textbf{u}nicorn, c\textbf{u}te   \\ \hline
\end{tabular}
\caption{\small Long and short vowels in English.\normalsize}
\label{tab:long_and_short_vowels_in_english}
\end{table}