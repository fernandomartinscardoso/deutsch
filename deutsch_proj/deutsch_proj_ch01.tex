%Chapter 1

\fancyhead[RO]{\slshape Aussprachekurs}  % Mark on right odd pages
\fancyhead[LE]{\slshape Deutschkurs}     % Mark on left even pages
\chapter*{Aussprachekurs}
\addcontentsline{toc}{chapter}{Aussprachekurs}
\label{chap:aussprachekurs}

\begin{flushright}
\mbox{%
\begin{minipage}{0.5\textwidth}
{\textit{\footnotesize{Der Alte verliert eines der größten Menschenrechte: \\er wird nicht mehr von seines Gleichen beurteilt.}}}
\begin{flushright}
{\scriptsize{Johann Wolfgang von Goethe}}
\end{flushright}
\end{minipage}%
}
\end{flushright}

This chapter is dedicated to study the German language pronunciation. The content is based on the first chapter of the Berlitz book \cite{berlitz2000}, on the Aussprachekurs from Professor Raville \cite{raville2025}, and on the pronunciation indicated in Langenscheidt Dictionary \cite{langen2015} and Wiktionary \cite{wiktionary} using International Phonetic Alphabet (IPA).

\section*{Accent and pronunciation}
\addcontentsline{toc}{section}{Accent and pronunciation}
\label{sec:accent_and_pronunciation}

According to Professor Raville \cite{raville2025}, accent is a particular way of pronouncing certain phonemes, which can change the ``melody'' and ``rhythm'' of a particular word or phrase. While pronunciation has a more rigid structure which, even with the variation in accents, must be preserved so as not to compromise the communication process. Therefore, this material focus on German standard pronunciation (\textit{Standardaussprache}) to keep the speech clear during verbal communication.

To check the stressed\footnote{In linguistics, and particularly phonology, stress or accent is the relative emphasis or prominence given to a certain syllable in a word or to a certain word in a phrase or sentence \cite{wiki_stress}. But this accent must not be confused with the sociolinguistic meaning of accent mentioned on the previous paragraph, which is a way of pronouncing a language that is distinctive to a country, area, social class, or individual \cite{wiki_accent}.} syllable of German words, the website of the famous German dictionary Duden can be consulted at the following link \url{https://www.duden.de/}. To check phonetic transcription and etimology, Wiktionary, the free dictionary available in \url{https://en.wiktionary.org/wiki/} has an excellent database. Finally, to check pronunciation with audio recordings of native speakers, the website \url{https://www.forvo.com/} is a great option.

\section*{How to improve the pronunciation in German}
\addcontentsline{toc}{section}{How to improve the pronunciation in German}
\label{sec:improve_pronunciation}

Here are recommended tasks for pronunciation improvement \cite{raville2025}:
\begin{itemize}
    \item Always read aloud, and use a voice recorder frequently.
    \item Check your pronunciation on Google Translate or similar online translators. Do they ``understand'' you?
    \item Identify what is most challenging for you to pronounce and practice a lot until you master it.
    \item When speaking German, articulate a lot and exaggerate. This is normal in the beginning.
    \item Sing in German.
\end{itemize}

\section*{Das Deutsche Alphabet (The German Alphabet)}
\addcontentsline{toc}{section}{Das Deutsche Alphabet (The German Alphabet)}
\label{sec:german_alphabet}

Before diving into vocabulary and grammar, establishing a solid foundation in pronunciation is crucial. The sounds of a language are its DNA, and mastering them early will significantly accelerate your progress and confidence.

The German alphabet is, thankfully, very similar to the English (Latin) alphabet, sharing the same 26 letters. However, the German language incorporates four additional characters that are essential for accurate reading and speaking: the three mutated vowels (Umlauts) and the sharp S (\textit{Eszett} or \textit{scharfes S}).

This section will provide a comprehensive overview of the following components:

\begin{enumerate}
    \item The 26 standard letters and how their names and associated sounds differ from their English counterparts.
    \item The four special characters (Ä, Ö, Ü, and ß) and their distinct phonetic values.
    \item Common diphthongs and letter combinations (like ei, eu, ie, ch, and sch) that follow predictable rules and are key to proper German articulation.
\end{enumerate}

Knowing how to correctly name and sound out each letter will unlock the ability to spell German words clearly and, most importantly, pronounce them correctly.

\subsection*{A}
\addcontentsline{toc}{subsection}{A}
\label{ssec:letter_a}

Letter name \textipa{[a]}. This letter sounds like a as in \textbf{a}lgae.

Examples:
\begin{enumerate}
    \item \textbf{aus}: \textipa{[aUs]}
    \item \textbf{auf}: \textipa{[aUf]}
    \item \textbf{an}: \textipa{[an]}
    \item \textbf{aktuell}: \textipa{[aktu"El]}
\end{enumerate}

\subsection*{Ä}
\addcontentsline{toc}{subsection}{Ä}
\label{ssec:letter_a_umlaut}

Letter name \textipa{[a]} Umlaut, pronounced \textipa{[E:]}. This letter may sound long (i.e., stressed), or short (i.e., unstressed) \cite{umlaute2021}.

Examples of long ä:
\begin{enumerate}    
    \item \textbf{Hähnchen}: \textipa{["hE:n\c{c}@n]}
    \item \textbf{Käse}: \textipa{["kE:z@]}
    \item \textbf{schläfst}: \textipa{["SlE:fst]}
    \item \textbf{Verspätung}: \textipa{[fEr"SpE:tUN]}
\end{enumerate}

Examples of short ä:
\begin{enumerate}
    \item \textbf{ändern}: \textipa{["End@rn]}
    \item \textbf{Gäste}: \textipa{["gEst@]}
    \item \textbf{Männer}: \textipa{["mEn@r]}
    \item \textbf{März}: \textipa{[mErts]}
    \item \textbf{wäscht}: \textipa{[vESt]}
\end{enumerate}

Depending on the area of Germany, ä may sound short in words where it is usually long, e.g., \textbf{schläfst} being pronounced as \textipa{["SlEfst]}, and \textbf{später} being pronounced \textipa{["SpEt@r]}.

\subsection*{B}
\addcontentsline{toc}{subsection}{B}
\label{ssec:letter_b}

Letter name \textipa{[be:]}. This letter sounds like b in open syllables, as in \textbf{b}ow. And it sounds like silent p at the end of words or in closed syllables, as in ma\textbf{p}.

Examples:
\begin{enumerate}
    \item \textbf{Ab}: \textipa{[ap]}
    \item \textbf{bald}: \textipa{[balt]}
    \item \textbf{bekommen}: \textipa{[be"kOm@n]}
    \item \textbf{Bier}: \textipa{[bi:r]}
    \item \textbf{Bus}: \textipa{[bUs]}
    \item \textbf{gelb}: \textipa{[gElp]}
    \item \textbf{gibt}: \textipa{[gIpt]}
    \item \textbf{habt}: \textipa{[hapt]}
    \item \textbf{halb}: \textipa{[halp]}
    \item \textbf{Obst}: \textipa{[o:pst]}
    \item \textbf{siebzehn}: \textipa{["zi:ptse:n]}
\end{enumerate}

\subsection*{C}
\addcontentsline{toc}{subsection}{C}
\label{ssec:letter_c}

Letter name \textipa{[tse:]}. This letter sounds like ts before e and i, and sounds like k before a, o and u. It is not a common letter in German language, mostly used in foreign words incorporated into German.

Examples:
\begin{enumerate}
    \item \textbf{Café}: \textipa{[ka"fe:]}
    \item \textbf{campen}: \textipa{["kEmp@n]}
    \item \textbf{Celsius}: \textipa{["tselziUs]}
    \item \textbf{Chaos}: \textipa{["ka:os]}
    \item \textbf{Curry}: \textipa{["k{\oe}ri]}
\end{enumerate}

\subsection*{D}
\addcontentsline{toc}{subsection}{D}
\label{ssec:letter_d}

Letter name \textipa{[de:]}. This letter sounds like d in open syllables, as in \textbf{d}og. And it sounds like silent t at the end of words or in closed syllables, as in ca\textbf{t}.

Examples:
\begin{enumerate}
    \item \textbf{Bild}: \textipa{[bIlt]}
    \item \textbf{Dame}: \textipa{["da:m@]}
    \item \textbf{dämpfen}: \textipa{["dEmpf@n]}
    \item \textbf{davor}: \textipa{["da:fOr,da"fo:r]}
    \item \textbf{Freund}: \textipa{[frOYnt]}
    \item \textbf{Hand}: \textipa{[hant]}
    \item \textbf{Kind}: \textipa{[kint]}
    \item \textbf{Land}: \textipa{[lant]}
    \item \textbf{Stadt}: \textipa{[Stat]}
    \item \textbf{Versand}: \textipa{[fEr"zant]}
\end{enumerate}

\subsection*{E}
\addcontentsline{toc}{subsection}{E}
\label{ssec:letter_e}

Letter name \textipa{[e:]}. This letter sounds like Spanish \textbf{e}, as in abu\textbf{e}lo. But it sounds more subtle at the end of words, as in mom\textbf{e}nt.

Examples:
\begin{enumerate}
    \item \textbf{eine}: \textipa{["aIn@]}
    \item \textbf{esse}: \textipa{["Es@]}
    \item \textbf{Frage}: \textipa{["fra:g@]}
    \item \textbf{heute}: \textipa{["hOYt@]}
    \item \textbf{lese}: \textipa{["le:s@]}
    \item \textbf{Sprache}: \textipa{["Spra:x@]}
    \item \textbf{Wange}: \textipa{["vang@]}
\end{enumerate}

\subsection*{F}
\addcontentsline{toc}{subsection}{F}
\label{ssec:letter_f}

Letter name \textipa{[Ef]}. In words, this letter sounds like f as in \textbf{f}ate or \textbf{f}riend.

Examples:
\begin{enumerate}
    \item \textbf{Fach}: \textipa{[fax]}
    \item \textbf{fegen}: \textipa{["fe:g@n]}
    \item \textbf{Feier}: \textipa{["faI@r]}
    \item \textbf{Flug}: \textipa{[flu:k]}
    \item \textbf{freundlich}: \textipa{["frOYntlI\c{c}]}
    \item \textbf{Frucht}: \textipa{[frUxt]}
    \item \textbf{Fußball}: \textipa{["fu:sbal]}
\end{enumerate}

\subsection*{G}
\addcontentsline{toc}{subsection}{G}
\label{ssec:letter_g}

Letter name \textipa{[ge:]}. This letter sounds like g in open syllables, as in \textbf{g}ate. And it sounds like silent k at the end of words or closed syllables, as in dar\textbf{k}.

Examples:
\begin{enumerate}
    \item \textbf{Flugzeug}: \textipa{["flu:ktsOYk]}
    \item \textbf{Gesicht}: \textipa{[g@"zI\c{c}t]}
    \item \textbf{lügt}: \textipa{[ly:kt]}
    \item \textbf{mag}: \textipa{[ma:k]}
    \item \textbf{sagt}: \textipa{[za:kt]}
    \item \textbf{Sonntag}: \textipa{["zOnta:k]}
    \item \textbf{Weg}: \textipa{[ve:k]}
\end{enumerate}

As per article in \cite{duden2025}, there are two exceptions for the general rule of g pronunciation: the g is not pronounced after n, and it is pronounced as ch in the German word \textbf{ich} when it comes after i.

Examples:
\begin{enumerate}
    \item \textbf{Hunger}: \textipa{["hUN@r]}
    \item \textbf{lang}: \textipa{[laN]}
    \item \textbf{länger}: \textipa{["lEN@r]}
    \item \textbf{Zeitung}: \textipa{["tsaItUN]}
    \item \textbf{vergänglich}: \textipa{[fEr"gENlI\c{c}]}
    \item \textbf{ewig}: \textipa{["e:vI\c{c}]}
    \item \textbf{fähig}: \textipa{["fE:I\c{c}]}
    \item \textbf{fertig}: \textipa{["fErtI\c{c}]}
    \item \textbf{richtig}: \textipa{["ri\c{c}tI\c{c}]}
    \item \textbf{ständig}: \textipa{["StEndI\c{c}]}
    \item \textbf{vierzig}: \textipa{["fi:rtsI\c{c}]}
\end{enumerate}

\subsection*{H}
\addcontentsline{toc}{subsection}{H}
\label{ssec:letter_h}

Letter name \textipa{[ha:]}. It sounds like h as in \textbf{h}ave at the beginning of words, it is not pronounced between vowels, and it prolongs the duration of the preceding vowel.

Examples of h at the beginning of words:
\begin{enumerate}
    \item \textbf{haben}: \textipa{["ha:b@n]}
    \item \textbf{Hallo}: \textipa{[ha"lo:]}
    \item \textbf{Haus}: \textipa{[haUs]}
    \item \textbf{heißen}: \textipa{["haIs@n]}
    \item \textbf{Humor}: \textipa{[hu"mo:r]}
\end{enumerate}

Examples of h between vowels and prolonging vowels:
\begin{enumerate}
    \item \textbf{früher}: \textipa{["fry:@r]}
    \item \textbf{gehen}: \textipa{["ge:@n]}
    \item \textbf{Höhe}: \textipa{["h{\o}:@]}
    \item \textbf{ihr}: \textipa{[i:r]}
    \item \textbf{Uhr}: \textipa{[u:r]}
\end{enumerate}

Exceptions: compound words, e.g., \textbf{woher}, and foreign words adapted to German, e.g., \textbf{Alkohol}.

\subsection*{I}
\addcontentsline{toc}{subsection}{I}
\label{ssec:letter_i}

Letter name \textipa{[i:]}. It sounds like i in \textbf{i}llness.

\subsection*{J}
\addcontentsline{toc}{subsection}{J}
\label{ssec:letter_j}

Letter name \textipa{[jOt]}. It sounds like i as in man\textbf{i}a.

Examples:
\begin{enumerate}
    \item \textbf{ja}: \textipa{[ja:]}
    \item \textbf{Jahr}: \textipa{[ja:r]}
    \item \textbf{jetzt}: \textipa{[jEtst]}
    \item \textbf{jedoch}: \textipa{[je:"dOx]}
    \item \textbf{jemand}: \textipa{["je:mant]}
    \item \textbf{jetzt}: \textipa{[jEtst]}
    \item \textbf{Junge}: \textipa{["juN@]}
\end{enumerate}

In words of English origin, the original pronunciation of j is kept as in \textbf{j}ob. Examples: \textbf{joggen} (\textipa{["{\textdyoghlig}Og@n]}), and \textbf{Pyjama} (\textipa{[py"{\textdyoghlig}a:ma]}).

\subsection*{K}
\addcontentsline{toc}{subsection}{K}
\label{ssec:letter_k}

Letter name \textipa{[ka:]}. It sounds like k as in \textbf{k}id, with a plosive intonation \cite{hessen2008}.

\subsection*{L}
\addcontentsline{toc}{subsection}{L}
\label{ssec:letter_l}

Letter name \textipa{[El]}. In closed syllables\cite{syllables}, this letter sounds like the Spanish l as in mie\textbf{l}.

Examples:
\begin{enumerate}
    \item \textbf{Alkohol}: \textipa{["alkoho:l]}
    \item \textbf{helfen}: \textipa{["hElf@n]}
    \item \textbf{Himmel}: \textipa{["hIm@l]}
    \item \textbf{Hotel}: \textipa{[ho"tEl]}
    \item \textbf{Milch}: \textipa{[mil\c{c}]}
    \item \textbf{schnell}: \textipa{[SnEl]}
\end{enumerate}

\subsection*{M}
\addcontentsline{toc}{subsection}{M}
\label{ssec:letter_m}

Letter name \textipa{[Em]}. At the end of words, it sounds like the letter m as in roo\textbf{m}.

Examples:
\begin{enumerate}
    \item \textbf{Baum}: \textipa{[baUm]}
    \item \textbf{einem}: \textipa{["aIn@m]}
    \item \textbf{im}: \textipa{[Im]}
    \item \textbf{komm}: \textipa{[kOm]}
    \item \textbf{Raum}: \textipa{[raUm]}
    \item \textbf{wem}: \textipa{[ve:m]}
\end{enumerate}

\subsection*{N}
\addcontentsline{toc}{subsection}{N}
\label{ssec:letter_n}

Letter name \textipa{[En]}. At the end of words, this letter sounds like n as in Heave\textbf{n}.

Examples:
\begin{enumerate}
    \item \textbf{Bösen}: \textipa{["b{\o}:z@n]}
    \item \textbf{Mädchen}: \textipa{["mE:t\c{c}@n]}
    \item \textbf{mein}: \textipa{[maIn]}
\end{enumerate}

\subsection*{O}
\addcontentsline{toc}{subsection}{O}
\label{ssec:letter_o}

Letter name \textipa{[o:]}. This letter sounds like a long o as in Spanish word p\textbf{o}llo, or short o as in h\textbf{o}t.

Examples:
\begin{enumerate}
    \item \textbf{rot}: \textipa{[ro:t]}
    \item \textbf{Socke}: \textipa{["zOk@]}
    \item \textbf{Soda}: \textipa{["zo:da]}
    \item \textbf{Tochter}: \textipa{["tOxt@r]}
\end{enumerate}

\subsection*{P}
\addcontentsline{toc}{subsection}{P}
\label{ssec:letter_p}

Letter name \textipa{[pe:]}. This letter sounds like p as in \textbf{p}eople, often with plosive intonation.

Examples:
\begin{enumerate}
    \item \textbf{Palast}: \textipa{[pa"last]}
    \item \textbf{Pferd}: \textipa{[pfe:rt]}
    \item \textbf{plus}: \textipa{[plUs]}
    \item \textbf{Prinz}: \textipa{[prInts]}
\end{enumerate}

\subsection*{Q}
\addcontentsline{toc}{subsection}{Q}
\label{ssec:letter_q}

Letter name \textipa{[ku:]}. This letter sounds like k as in \textbf{k}ite, and it is always followed by the letter u, forming the sound \textipa{[kv]}.

Examples:
\begin{enumerate}
    \item \textbf{Qual}: \textipa{[kva:l]}
    \item \textbf{Qualität}: \textipa{[kvali"tE:t]}
    \item \textbf{Quelle}: \textipa{["kvEl@]}
    \item \textbf{Quiz}: \textipa{[kvIs]}
\end{enumerate}

\subsection*{R}
\addcontentsline{toc}{subsection}{R}
\label{ssec:letter_r}

Letter name \textipa{[Er]}. This letter sounds like a guttural r, pronounced at the back of the throat (\textit{Reibe-R}). At the end of words or in closed syllables\footnote{Closed syllables have coda consonants. A coda is the consonant sound(s) that appear(s) at the end of a syllable, following the vowel (the nucleus). For example, in the word ``print'', the consonants ``nt'' form the coda, while the vowel ``i'' is the nucleus and ``pr'' is the onset. \url{https://psychologyoflanguage.pressbooks.tru.ca/chapter/syllables/} and \url{https://simple.wikipedia.org/wiki/Syllable}} , it sounds like a schwa \textipa{[@]}\footnote{ The "lazy" or "reduced" sound that appears in unstressed syllables, like a as in w\textbf{a}nn\textbf{a}.}.

Examples:
\begin{enumerate}
    \item \textbf{der}: \textipa{[dEr]}
    \item \textbf{Erde}: \textipa{["E:rd@]}
    \item \textbf{erledigen}: \textipa{[Er"le:dIg@n]}
    \item \textbf{Humor}: \textipa{[hu"mo:r]}
    \item \textbf{Kraut}: \textipa{[kraUt]}
    \item \textbf{leer}: \textipa{[le:r]}
    \item \textbf{lernen}: \textipa{["lE:rn@n]}
    \item \textbf{rotieren}: \textipa{[ro"ti:r@n]}
    \item \textbf{Rabatt}: \textipa{[ra"bat]}
    \item \textbf{sehr}: \textipa{[ze:r]}
    \item \textbf{warm}: \textipa{[varm]}
    \item \textbf{wer}: \textipa{[ve:r]}
    \item \textbf{wir}: \textipa{[vi:r]}
\end{enumerate}

\subsection*{S}
\addcontentsline{toc}{subsection}{S}
\label{ssec:letter_s}

Letter name \textipa{[Es]}. This letter sounds like English z in open syllables as in \textbf{z}ebra. Before p and t, usually at the beginning of words, it sounds like sh as in fi\textbf{sh}. For further cases in closed syllables, it sounds like s as in fa\textbf{s}t.

Examples:
\begin{enumerate}
    \item \textbf{sagen}: \textipa{["zag@n]}
    \item \textbf{sie}: \textipa{[zi:]}
    \item \textbf{Sache}: \textipa{["zax@]}
    \item \textbf{Suppe}: \textipa{["zUp@]}
    \item \textbf{Sport}: \textipa{["Spo:rt]}
    \item \textbf{Spanisch}: \textipa{["Spa:niS]}
    \item \textbf{Stunde}: \textipa{["Stund@]}
    \item \textbf{Stadt}: \textipa{[Stat]}
    \item \textbf{Lust}: \textipa{[lUst]}
    \item \textbf{Ost}: \textipa{[Ost]}
\end{enumerate}

\subsection*{T}
\addcontentsline{toc}{subsection}{T}
\label{ssec:letter_t}

Letter name \textipa{[te:]}. This letter sounds like t as in \textbf{t}op, with a plosive intonation.

\subsection*{U}
\addcontentsline{toc}{subsection}{U}
\label{ssec:letter_u}

Letter name \textipa{[u:]}. This letter sounds similar to English oo as in tr\textbf{oo}per.

Examples:
\begin{enumerate}
    \item \textbf{Kultur}: \textipa{[kUl"tu:r]}
    \item \textbf{Mund}: \textipa{[mUnt]}
    \item \textbf{unter}: \textipa{["Unt@r]}
\end{enumerate}

\subsection*{V}
\addcontentsline{toc}{subsection}{V}
\label{ssec:letter_v}

Letter name \textipa{[faU]}. This letter sounds like f as in \textbf{f}ade.

Examples:
\begin{enumerate}
    \item \textbf{Vater}: \textipa{["fa:t@r]}
    \item \textbf{vier}: \textipa{["fi:@r]}
    \item \textbf{Volk}: \textipa{[fOlk]}
\end{enumerate}

\subsection*{W}
\addcontentsline{toc}{subsection}{W}
\label{ssec:letter_w}

Letter name \textipa{[ve:]}. This letter sounds like v as in \textbf{v}ase.

Examples:
\begin{enumerate}
    \item \textbf{Wasser}: \textipa{["vas@r]}
    \item \textbf{wegen}: \textipa{["ve:g@n]}
    \item \textbf{weiß}: \textipa{[vaIs]}
\end{enumerate}

\subsection*{X}
\addcontentsline{toc}{subsection}{X}
\label{ssec:letter_x}

Letter name \textipa{[Iks]}. This letter sounds like x as in bo\textbf{x}.

Examples:
\begin{enumerate}
    \item \textbf{extrem}: \textipa{[Eks"tre:m]}
    \item \textbf{Praxis}: \textipa{["praksIs]}
    \item \textbf{Text}: \textipa{[tEkst]}
\end{enumerate}

\subsection*{Y}
\addcontentsline{toc}{subsection}{Y}
\label{ssec:letter_y}

Letter name \textipa{["ypsIlOn]}. This letter sounds like ü as in \textbf{ü}brig.

Examples:
\begin{enumerate}
    \item \textbf{analysieren}: \textipa{[analy"zi:r@n]}
    \item \textbf{sympathisch}: \textipa{[zym"patIS]}
    \item \textbf{System}: \textipa{[zys"te:m]}
\end{enumerate}

\subsection*{Z}
\addcontentsline{toc}{subsection}{Z}
\label{ssec:letter_z}

Letter name \textipa{[tsEt]}. This letter sounds like z as in Pi\textbf{zz}a (always ts sound for German words). 

Examples:
\begin{enumerate}
    \item \textbf{jetzt}: \textipa{[jEtst]}
    \item \textbf{spazieren}: \textipa{[Spa"tsi:r@n]}
    \item \textbf{Zimmer}: \textipa{["tsIm@r]}
\end{enumerate}

\subsection*{ß}
\addcontentsline{toc}{subsection}{ß}
\label{ssec:letter_eszett}

Letter name \textit{eszett} or \textit{scharfes S}. In words, this letter sounds like the ss in pa\textbf{ss}ing and comes after a long vowel or diphthong (blend of two vowel sounds in a single syllable).

According to \cite{wiki_germanalpha}, as the ß derives from a ligature of lower-case letters, it is itself exclusively lower-case. The proper transcription when it cannot be used, or when writing a word in all capital letters, is \textbf{ss} or \textbf{SS}. The ß is not used in Switzerland and Liechtenstein, where it was replaced by \textbf{ss}.

Examples:
\begin{enumerate}
    \item \textbf{außen}: \textipa{["aUs@n]}
    \item \textbf{dreißig}: \textipa{["draIsI\c{c}]}
    \item \textbf{Fuß}: \textipa{[fu:s]}
    \item \textbf{groß}: \textipa{[gro:s]}
    \item \textbf{Gruß}: \textipa{[gru:s]}
    \item \textbf{heißen}: \textipa{["haIs@n]}
    \item \textbf{schließen}: \textipa{["Sli:s@n]}
    \item \textbf{Spaß}: \textipa{[Spa:s]}
    \item \textbf{Straße}: \textipa{["Stra:s@]}
    \item \textbf{weiß}: \textipa{[vaIs]}
\end{enumerate}

When the vowel is short, the word is written with \textbf{ss}. Examples:

\begin{enumerate}
    \item \textbf{essen}: \textipa{["Es@n]}
    \item \textbf{Fluss}: \textipa{[flUs]}
    \item \textbf{gerissen}: \textipa{[g@"rIs@n]}
    \item \textbf{krass}: \textipa{[kras]}
    \item \textbf{muss}: \textipa{[mUs]}
\end{enumerate}
