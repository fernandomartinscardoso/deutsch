%Chapter 1

\fancyhead[RO]{\slshape Aussprachekurs}  % Mark on right odd pages
\fancyhead[LE]{\slshape Deutschkurs}     % Mark on left even pages
\chapter*{Aussprachekurs}
\addcontentsline{toc}{chapter}{Aussprachekurs}

\begin{flushright}
\mbox{%
\begin{minipage}{0.5\textwidth}
{\textit{\footnotesize{Der Alte verliert eines der größten Menschenrechte: \\er wird nicht mehr von seines Gleichen beurteilt.}}}
\begin{flushright}
{\scriptsize{Johann Wolfgang von Goethe}}
\end{flushright}
\end{minipage}%
}
\end{flushright}

This chapter is dedicated to study the German language pronunciation. The content is based on the first chapter of the book \cite{berlitz2000}, on the Aussprachekurs from Professor Raville in \cite{raville2025}, and on the pronunciation indicated in \cite{langen2015} using International Phonetic Alphabet (IPA).

\section*{Accent and pronunciation}

According to \cite{raville2025}, accent is a particular way of pronouncing certain phonemes, which can change the melody and rhythm of a particular word or phrase. While pronunciation has a more rigid structure which, even with the variation in accents, must be preserved so as not to compromise the communication process. Therefore, this material focus on German standard pronunciation (\textit{Standardaussprache}) to keep the speech clear during verbal communication.

To check the stressed\footnote{In linguistics, and particularly phonology, stress or accent is the relative emphasis or prominence given to a certain syllable in a word or to a certain word in a phrase or sentence \cite{wiki_stress}. But this accent must not be confused with the sociolinguistic meaning of accent mentioned on the previous paragraph, which is a way of pronouncing a language that is distinctive to a country, area, social class, or individual \cite{wiki_accent}.} syllable of German words, the website of the famous German dictionary Duden can be consulted at the following link \url{https://www.duden.de/}. However, to check pronunciation with audio recordings of native speakers, the website \url{https://www.forvo.com/} is a great option.

\section*{How to improve the pronunciation in German}

Quick check list from \cite{raville2025}:
\begin{itemize}
    \item Always read aloud.
    \item Use a voice recorder frequently.
    \item Check your pronunciation on Google Translate or similar online translators. Do they ``understand'' you?
    \item Identify what is most challenging for you to pronounce and practice a lot until you master it.
    \item When speaking German, articulate a lot and exaggerate. This is normal in the beginning.
    \item Sing in German.
\end{itemize}

\section*{A}

Letter name \textipa{[a]}. In words, this letter sounds like the a in \textbf{a}lgae.